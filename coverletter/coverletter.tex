%!TEX TS-program = xelatex
%!TEX encoding = UTF-8 Unicode
% Awesome CV LaTeX Template for Cover Letter
%
% This template has been downloaded from:
% https://github.com/posquit0/Awesome-CV
%
% Authors:
% Claud D. Park <posquit0.bj@gmail.com>
% Lars Richter <mail@ayeks.de>
%
% Template license:
% CC BY-SA 4.0 (https://creativecommons.org/licenses/by-sa/4.0/)
%


%-------------------------------------------------------------------------------
% CONFIGURATIONS
%-------------------------------------------------------------------------------
% A4 paper size by default, use 'letterpaper' for US letter
\documentclass[11pt, a4paper]{awesome-cv}

% Configure page margins with geometry
\geometry{left=1.4cm, top=.8cm, right=1.4cm, bottom=1.8cm, footskip=.5cm}

% Specify the location of the included fonts
\fontdir[fonts/]

% Color for highlights
% Awesome Colors: awesome-emerald, awesome-skyblue, awesome-red, awesome-pink, awesome-orange
%                 awesome-nephritis, awesome-concrete, awesome-darknight
\colorlet{awesome}{awesome-emerald}
% Uncomment if you would like to specify your own color
% \definecolor{awesome}{HTML}{CA63A8}

% Colors for text
% Uncomment if you would like to specify your own color
% \definecolor{darktext}{HTML}{414141}
% \definecolor{text}{HTML}{333333}
% \definecolor{graytext}{HTML}{5D5D5D}
% \definecolor{lighttext}{HTML}{999999}

% Set false if you don't want to highlight section with awesome color
\setbool{acvSectionColorHighlight}{true}

% If you would like to change the social information separator from a pipe (|) to something else
\renewcommand{\acvHeaderSocialSep}{\quad\textbar\quad}


%-------------------------------------------------------------------------------
%	PERSONAL INFORMATION
%	Comment any of the lines below if they are not required
%-------------------------------------------------------------------------------
% Available options: circle|rectangle,edge/noedge,left/right
\photo[circle,noedge,left]{profile}
\name{Vyas}{Ramasubramani}
\position{Ph.D. Candidate, University of Michigan}
\address{510 N. 4th Ave., Ann Arbor, MI, 48104}

\mobile{(+1) 408-421-2162}
\email{vyas.ramasubramani@gmail.com}
\homepage{vyasr.com}
\github{vyasr}
% \linkedin{vyas-ramasubramani}
% \gitlab{gitlab-id}
% \stackoverflow{SO-id}{SO-name}
% \twitter{@twit}
% \skype{skype-id}
% \reddit{reddit-id}
% \medium{madium-id}
% \googlescholar{googlescholar-id}{name-to-display}
%% \firstname and \lastname will be used
\googlescholar{vyLxpbkAAAAJ}{Vyas}
% \extrainfo{extra informations}

% \quote{``Be the change that you want to see in the world."}


%-------------------------------------------------------------------------------
%	LETTER INFORMATION
%	All of the below lines must be filled out
%-------------------------------------------------------------------------------
% The company being applied to
\recipient
  {Recruitment Team}
  {X Development}
% The date on the letter, default is the date of compilation
\letterdate{\today}
% The title of the letter
\lettertitle{Job Application for Computational Biologist}
% How the letter is opened
\letteropening{To Whom It May Concern:}
% How the letter is closed
\letterclosing{Sincerely,}
% Any enclosures with the letter
% \letterenclosure[Attached]{Curriculum Vitae}


%-------------------------------------------------------------------------------
\begin{document}

% Print the header with above personal informations
% Give optional argument to change alignment(C: center, L: left, R: right)
\makecvheader[R]

% Print the footer with 3 arguments(<left>, <center>, <right>)
% Leave any of these blank if they are not needed
\makecvfooter
  {\today}
  {Vyas Ramasubramani}
  {}

% Print the title with above letter informations
\makelettertitle

%-------------------------------------------------------------------------------
%	LETTER CONTENT
%-------------------------------------------------------------------------------
\begin{cvletter}
    The trajectory of my research has always circled biology.
    During my bachelor's, this research ranged from characterizing stem cell development in chicken embryos to analyzing bistability in metabolic networks.
    Over time, my work became more quantitative, including projects using microarray data to analyze the binding propensities of various zinc finger proteins and a molecular dynamics study of the conformational changes that occur when transcription factors stop translating along a DNA strand and begin to transcribe.
    My bachelor's thesis took a different turn, focusing on the creation of polymer particles for drug delivery purposes, and the synthesis of these ideas with my work on biomolecular modeling culminated in a PhD focused on a physics-based approach to solving bio-inspired colloidal materials problems.

    My research now involves extensive use of high performance computing, as well as a great deal of open-source software development designed to simplify and accelerate scientific discovery.
    I am a lead developer of signac and freud, two large open-source projects that have user bases extending far beyond my university, and I am also a core developer for HOOMD-blue, one of the most popular particle simulation toolkits.
    My work on these tools has been presented three times at the Scientific Computing in Python Conference, and a great deal of this code is accelerated using C++ and designed to work on both CPUs and GPUs.
    I am also comfortable working in other environments and with other languages; for instance, in my job as an analyst at the financial institution DC Energy prior to graduate school I also developed expertise in R and MySQL for large-scale analysis of power flow on the national electric grid.

    I am interested in continuing to combine these technical skills and scientific software development with an active research role applied to biological problems, and I believe that this position at X would be a great fit for these interests and my skill set.
    I am currently in the process of preparing most of my research work for various publications and I will soon be defending my PhD.
    I want to continue applying a developer's mindset to solving scientific problems, and I would love the opportunity to do so at The Moonshot Factory.
    Thank you for your consideration for this position, and I look forward to hearing from you.
\end{cvletter}


%-------------------------------------------------------------------------------
% Print the signature and enclosures with above letter informations
\makeletterclosing

\end{document}
