%-------------------------------------------------------------------------------
%	SECTION TITLE
%-------------------------------------------------------------------------------
\cvsection{Selected Projects}
%-------------------------------------------------------------------------------
%	CONTENT
%-------------------------------------------------------------------------------
\begin{cventries}
%---------------------------------------------------------
  \cventry
    {Lead Developer} % Role
    {signac framework ({\tiny github.com/glotzerlab/signac})} % Event
    {} % Location
    {} % Date(s)
    {
      \begin{cvitems} % Description(s)
        \item Co-led development of a NumFOCUS-affiliated project for data and workflow management in Python.
        \item Enabled flexible, easy-to-use storage mechanism for research data in computational science and machine learning.
        \item Drove development of a new, compact API for workflow definition.
      \end{cvitems}
    }
%---------------------------------------------------------
  \cventry
    {Lead Developer} % Role
    {freud ({\tiny github.com/glotzerlab/freud})} % Event
    {} % Location
    {} % Date(s)
    {
      \begin{cvitems} % Description(s)
        \item Co-led development of a C++ tool with Cython-generated Python bindings for fast, TBB-multithreaded analysis of molecular simulations.
        \item Spearheaded the creation of a standardized API achieving a high degree of flexibility and consistency.
        \item Achieved 5-15x performance improvements using more efficient algorithms and improved memory management as part of the 2.0 release.
      \end{cvitems}
    }
%---------------------------------------------------------
  \cventry
    {Core Developer} % Role
    {HOOMD-blue ({\tiny github.com/glotzerlab/hoomd-blue})} % Event
    {} % Location
    {} % Date(s)
    {
      \begin{cvitems} % Description(s)
        \item Made various contributions to a general purpose, high-performance particle simulation toolkit for GPU-accelerated molecular dynamics and Monte Carlo simulations that is currently one of NVIDIA's official HPC benchmarks.
        \item Enabled bidirectional MPI communication to enable multi-body force fields in MPI simulations.
        \item Implemented the Gilbert-Johnson-Keerthi algorithm on CPUs and GPUs as part of an anisotropic pair potential.
      \end{cvitems}
    }
\ifextended
%---------------------------------------------------------
  \cventry
    {Lead Developer} % Role
    {rowan ({\tiny github.com/glotzerlab/rowan})} % Event
    {} % Location
    {} % Date(s)
    {
      \begin{cvitems} % Description(s)
        \item Wrote a Python package for quaternion mathematics.
        \item Completely mirrored NumPy broadcasting syntax to enable high-performance operations on large arrays in pure Python.
      \end{cvitems}
    }
%---------------------------------------------------------
  \cventry
    {Lead Developer} % Role
    {coxeter ({\tiny github.com/glotzerlab/coxeter})} % Event
    {} % Location
    {} % Date(s)
    {
      \begin{cvitems} % Description(s)
        \item Wrote a Python package for geometric calculations, particularly focusing on polytopes in 2D and 3D.
        \item Developed and implemented a unified API for generating families of shapes via arbitrary parameterization.
      \end{cvitems}
    }
%---------------------------------------------------------
  \cventry
    {Contributor} % Role
    {gsd ({\tiny github.com/glotzerlab/gsd})} % Event
    {} % Location
    {} % Date(s)
    {
      \begin{cvitems} % Description(s)
        \item Contributed to the specification of a generic binary file format for particle simulation data and the implementing C code.
        \item Improved memory mapping usage to allow efficient read-only access to very large files.
      \end{cvitems}
    }
\fi
\ifoutdated
%---------------------------------------------------------
  \cventry
    {Co-creator} % Role
    {TigerChat} % Event
    {} % Location
    {} % Date(s)
    {
      \begin{cvitems} % Description(s)
      \item Worked under Professor Brian Kernighan to develop a platform for XMPP-based communication over CAS networks.
      \end{cvitems}
    }
\fi
\end{cventries}
